%\section{Drugi ekran}
%
%Second screen

\section{Tłumaczenia}

Pakiet beamer już od pewnego czasu używa narzędzia bardzo podobnego do narzędzi używanych w „profesjonalnych” pakietach oprogramowania, a służącego do łatwiejszego tworzenia dokumentów wielojęzycznych.

Pakiet nazywa się \href{https://ctan.org/pkg/translator}{translator}. Osobiście nie wiem, czy to dobry pomysł, czy zły, ale tak jest.

W szczególności korzystanie ze środowisk: \lstinline|example|, \lstinline|theorem|, \lstinline|definition|, czy poleceń \lstinline|\partpage|, \lstinline|\sectionpage|, \lstinline|\subsectionpage|,…

%
%\section{Narzędzia pomocnicze}
%
%\subsection{pdfpc}
%
%\url{https://pdfpc.github.io/}
%
%\subsection{pympress}
%
%\url{https://github.com/Cimbali/pympress}
%
%\subsection{PDF Presenter}
%
%\url{http://pdfpresenter.sourceforge.net/}
%


